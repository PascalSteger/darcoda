\section{Method}\label{sec:method}


The collisionless Boltzmann equation for a spherical system with
gravitational potential $\Phi$,

\begin{equation}
\frac{\text{d}f}{\text{d}t} = \frac{\partial f}{\partial t} + \nabla_{\vec{x}} f\cdot\vec{v} - \nabla_{\vec{v}} f\cdot\nabla_{\vec{x}}\Phi = 0,
\end{equation}

describes the motion of tracer stars with distribution function
$f(\vec{x},\vec{v}\,)$.


In spherical coordinates $(r, \theta, \phi)$, the collisionless
Boltzmann equation then reads as

\begin{equation}
\frac{\partial f}{\partial t} + \dot{r}\frac{\partial f}{\partial r} + \dot{\theta}\frac{\partial f}{\partial \theta} + \dot{\phi}\frac{\partial f}{\partial \phi} + \dot{v}_r\frac{\partial f}{\partial v_r}+\dot{v}_\theta\frac{\partial f}{\partial v_\theta} +\dot{v}_\phi\frac{\partial f}{\partial v_\phi} = 0
\end{equation}

with velocities

\begin{eqnarray}
\dot{r}       &=& v_r,\\
\dot{\theta}  &=& v_\theta/r\\
\dot{\phi}    &=& v_\phi / r \sin\theta.
\end{eqnarray}

The assumption of steady state hydrodynamic equilibrium gives
$\partial f/\partial t=0$ and $\bar{v}_r=0$, and using spherical
symmetry $\bar{v}_\theta=0$, $\bar{v}_\phi=0$, with a unique
tangential velocity dispersion
$\sigma_\phi^2=\sigma_\theta^2=\sigma_t^2$ yields

\begin{equation}\label{eq:Jeans}
\frac{1}{\nu}\frac{\partial}{\partial r}(\nu\sigma_{r}^2) + 2\frac{\sigma_{r}^2-\sigma_{t}^2}{r} = -\frac{\partial \Phi}{\partial r} = -\frac{GM(<r)}{r^2}
\end{equation}

with enclosed mass $M(<r)$, gravitational constant $G =
6.67398\cdot10^{-11} \text{m}^3/\text{kg}\,\text{s}^2$. The departure
from spherical hydrostatic equilibrium $\sigma_r^2=\sigma_t^2$ is
measured by the anisotropy parameter

\begin{equation}
\beta \equiv 1-\frac{\sigma_t^2}{\sigma_r^2}
\end{equation}

with values in the range from $-\infty$ (purely circular orbits)
through $0$ (hydrostatic equilibrium) to $1$ (purely radial orbits).

Integrating both sides of equation \ref{eq:Jeans} gives the main
equation of this paper,

\begin{equation}\label{eq:main}
\sigma_r^2(R) = \frac{1}{\nu(R)}\exp\left(-2\int_{r_{min}}^{r}\frac{\beta(s)}{s}\text{d}s\right)\cdot\qquad
\end{equation}
\begin{equation*}
\qquad\int_R^\infty \frac{GM(r)\nu(r)}{r^2} \exp\left(2\int_{r_{min}}^r\frac{\beta(s)}{s}\text{d}s\right)\text{d}r.
\end{equation*}

For distant spherical systems, only the projected velocity dispersion
$\siglos$ can be measured, which in our case is given by

\begin{equation}\label{eq:LOS}
\siglos^2(R) = \frac{2}{\Sigma(R)}\int_R^\infty \left(1-\beta\frac{R^2}{r^2}\right) \frac{\nu(r)\sigma_r^2(r) r}{\sqrt{r^2-R^2}}\text{d}r,
\end{equation}

where $\Sigma(R)$ denotes the surface mass density at radius $R$.

In the following, we present a fully non-parametric method for the
solution of equation \ref{eq:LOS} for the total gravitating mass
density $\rho(r)$, given observed $\nu(r)$ and $\siglos(r)$, where $r$
denotes the projected two-dimensional radius from the center of mass
of the spherical system. We get the enclosed mass $M(<r)$ from the
density via

\begin{equation}
M(<r) = \int_0^r \rho(r) r^2 dr,
\end{equation}

which shows up in eq. \ref{eq:main}. In principle, the above-mentioned
method can be generalized to investigate alternative gravity models,
if the acceleration $GM(r)/r^2$ is replaced with the respective form
of $-\partial\Phi/\partial r$.

The degeneracy between mass $M$ and velocity anisotropy $\beta$ is
accounted for: For any non-isothermal system, we let vary the
anisotropy $\beta(r)$ as well. We checked that in the case of a simple
Hernquist profile, $\beta(r)\approx0$ is retrieved correctly.

The main idea is to let an Monte Carlo Markov Chain sample the
parameter space $[\nu_i, \siglosi, \rho]$ for distinct populations
$i=1...N$ of stellar or gaseous tracers.

Two different approaches are taken into account for the sampling of
densities:

Tracer densities $\nu_i$ are expected to fall with increasing
radii. To ensure this, one can explicitly build a monotonic function

\begin{equation}
\nu_i(r) = \int_0^r \tilde{\nu}_i(r')dr'
\end{equation}

with parameters $\tilde{\nu}_i(r')>0$ discretized in bins. The other
possibility is to let $\tilde{\nu}_i(r')$ vary freely, or just sample
$\nu_i(r')=\tilde{\nu}_i|_{r'}$ directly. As it turns out, the MCMC
prefers decreasing densities anyhow, so the $\nu_i$ parameters were
allowed to vary within the priors described below.

Furthermore, the sampling might be done in a linear or logarithmic
fashion. Wherever negative components are required, $\beta_i(r)$,
linear sampling was chosen, s.t.

\begin{equation}
\beta_i^{(n+1)}(r) = \beta_i^{(n)}(r) + \delta\beta_i(r)
\end{equation}

with new parameter $\beta_i^{(n+1)}(r)$ in iteration $n+1$ determined
from its old value $\beta_i^{(n)}(r)$ at iteration $n$ and the
parameter stepsize $\delta\beta_i(r)$ drawn from a random uniform
distribution. On the other hand, for any positive definite parameter
that needs to span a range in logarithmic space, $\nu_i(r)$ and
$\rho(r)$, we sample logarithmically,

\begin{equation}
\nu_i^{(n+1)}(r) = 10^{\tilde{\nu}_i^{(n+1)}},\qquad \tilde{\nu}_i^{(n+1)}(r) = \tilde{\nu}_i^{(n)}(r) + \delta\tilde{\nu}_i(r)
\end{equation}

In a next step, $\siglosi(r)$ is calculated from $\nu_i$, $\rho(r)$, and
$\beta_i(r)$ according eq. \ref{eq:LOS}. This is done numerically,
involving three integrations, which are performed with polynomial extrapolations
of the integrands up to infinity, s.t. contributions from
$\rho(r>r_{max}$ hinders an artificial falloff of
$\siglos$. The additional parameter of the slope is calculated from the second 
half of all bins.

The last step involves comparison of the projected $\nu_i(r)$, $\sigma_i(r)$ and
$beta_i(r)$, if available, to the data for the tracer populations to
get an error function

\begin{eqnarray}
\chi^2 &=& \sum_{i=1}^N \chi_{\nu,i}^2 + \chi_{\sigma,i}^2 + \chi_{\beta,i}^2\\
\chi_{\nu,i}^2 &=& \sum_{j=1}^{N_{bin}} \left(\frac{\nu_{i,\text{data}}(r_j)-\nu_{i,\text{model}}(r_j)}{\epsilon_\nu(r_j)}\right)^2
\end{eqnarray}

and accordingly for $\chi_{\sigma,i}^2$ and $\chi_{\beta,i}^2$. In
absence of a measured $\beta_i(r)$, we set $\chi_{\beta,i}^2=0$.

The model for iteration $n+1$ is accepted if its close or below the previous iteration,

\begin{equation}
\exp(\chi^2_n - \chi^2_{n+1}) < \varepsilon,\qquad \varepsilon\in [0,1)
\end{equation}

for a uniform random $\varepsilon$. Otherwise the model is rejected.

The stepsize can vary for each bin, and is changed during an
initialization phase. If the acceptance rate of models lies between
0.24 and 0.26, it is decreased by factors of 1.01, else, it is
increased by the same amount. After a burn-in phase of several 100
accepted models with $\chi^2<\chi_{end}^2 = 70$, the stepsize is
frozen and the MCMC starts storing the accepted models for further
statistical analysis. A default $10^5$ models are taken, where nothing
else is indicated.



\subsection{Binning characteristics}

The number of bins for $\nu, \sigma, \beta, \rho$ are free
parameters. They are set to fulfill

\begin{equation}
n_\nu = n_\sigma = n_\beta = n_\rho \leq n_{\text{data}}
\end{equation}

with number of datapoints $n_{\text{data}}$. This choice simplifies
integration greatly, and prevents invention of information on scales
smaller than the frequency of datapoints.

\TODO{check that nbin is set s.t. }

\begin{equation}
\chi^2_{red} = \frac{\chi^2}{n_{\text{data}} - n_\nu - n_\sigma - n_\beta - n_\rho -1}
\end{equation}

is minimized, and still the whole parameter space is tracked.

The whole parameter space for $\beta_i$ is sampled with $n_\beta=12$
if $n_{\text{data}}=30$ in the case of 10000 tracers.

The dark matter density is calculated by subtracting the measured
baryon density from the dynamical mass density.

\subsection{Priors}
Following priors are included in the MCMC, and can help to reject
unphysical models from the start:

\begin{enumerate}
\item[1)] cprior: $M(r=0) = 0$, the central mass is set to 0;

\item[2)] bprior: $\rho(r) \geq \rho_b(r)-\epsilon_{\rho,b}(r) \forall
  r\geq0$, ensures that no models with overall densities below the
  measured baryon density (reduced by the measurement error) are
  considered any further;

\item[3)] lbprior: $M(r>r_{max}) \leq M(<r_{max})/3.$, rejects any
  model which has more than $33\%$ of the overall mass up to the
  outermost radius in the extrapolated bins;

\item[4)] rising $\rho$ prior: $(\rho(r+\Delta r)-\rho(r))/\rho(r)\leq
  0.5$, prevents $\rho$ rising more than $50\%$ for the next
  bin. There is no reason for the overall mass density to rise
  outwards in dynamically old systems. It might be favourable for
  convergence, though, if a dip in $\rho(r)$ does not lead to
  immediate rejection of all models with correct $\rho(r+\Delta r)$;

\item[5)] $\beta_i(r+\Delta r)-\beta_i(r) < 0.5$: prevent any sudden
  jumps in $\beta_i$;
\end{enumerate}

We show in the appendix what effects the disabling of some of these
priors have.
