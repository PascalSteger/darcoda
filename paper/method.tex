\section{Method}\label{sec:method}
The collisionless Boltzmann equation for a spherical system with
gravitational potential $\Phi$,

\begin{equation}
\frac{\text{d}f}{\text{d}t} = \frac{\partial f}{\partial t} + \nabla_{\vec{x}} f\cdot\vec{v} - \nabla_{\vec{v}} f\cdot\nabla_{\vec{x}}\Phi = 0,
\end{equation}

describes the motion of tracer stars at positions $\vec{x}$, with
velocity $\vec{v}$, and with distribution function
$f(\vec{x},\vec{v}\,)$.

In spherical coordinates $(r, \theta, \phi)$, the collisionless
Boltzmann equation then reads as

\begin{equation}
    \frac{\partial f}{\partial t} + \dot{r}\frac{\partial f}{\partial r} + \dot{\theta}\frac{\partial f}{\partial \theta} + \dot{\phi}\frac{\partial f}{\partial \phi} + \dot{v}_r\frac{\partial f}{\partial v_r}+\dot{v}_\theta\frac{\partial f}{\partial v_\theta} +\dot{v}_\phi\frac{\partial f}{\partial v_\phi} = 0
\end{equation}

with velocities

\begin{eqnarray}
\dot{r}       &=& v_r,\\
\dot{\theta}  &=& v_\theta/r\\
\dot{\phi}    &=& v_\phi / r \sin\theta.
\end{eqnarray}

The assumption of steady state hydrodynamic equilibrium gives
$\partial f/\partial t=0$ and $\widebar{v}_r=0$, and using spherical
symmetry $\bar{v}_\theta=0$, $\bar{v}_\phi=0$ with a unique tangential
velocity dispersion $\sigma_\phi^2=\sigma_\theta^2=\sigma_t^2$ and a
fourth order moment $\widebar{v_r^4}$ we get

\begin{eqnarray}\label{eq:Jeans}
    \frac{1}{\nu}\frac{\partial}{\partial r}(\nu\sigma_{r}^2) + 2\frac{\sigma_{r}^2-\sigma_{t}^2}{r} = -\frac{\partial \Phi}{\partial r} &=& -\frac{GM(<r)}{r^2},\\
    \frac{\partial}{\partial r}(\nu\bar{v_r^4})+\frac{2\beta}{r}\nu\bar{v_r^4}+3\nu\sigma_r^2\frac{\partial\Phi}{\partial r}&=&0,
\end{eqnarray}

with enclosed total mass $M(<r)$ inside radius $r$, density $\nu$ of
any collection of tracer particles in the potential $\Phi$, and
gravitational constant $G = 6.67398\cdot10^{-11}
\text{m}^3/\text{kg}\,\text{s}^2$. The departure from spherical
hydrostatic equilibrium $\sigma_r^2=\sigma_t^2$ is measured by the
anisotropy parameter

\begin{equation}
\beta \equiv 1-\frac{\sigma_t^2}{\sigma_r^2}
\end{equation}

with values in the range from $-\infty$ (purely circular orbits)
through $0$ (hydrostatic equilibrium) to $1$ (purely radial orbits).

Integrating both sides of equation \ref{eq:Jeans} gives the radial
velocity dispersion as function of the projected radius $R$, and
correspondingly the fourth order moment,

\begin{equation}\label{eq:main}
    \sigma_r^2(R) = \frac{1}{\nu(R)}\exp\left(-2\int_{r_{min}}^{R}\frac{\beta(s)}{s}\text{d}s\right)\cdot\qquad
\end{equation}
\begin{equation*}
    \qquad\int_R^\infty \frac{GM(r)\nu(r)}{r^2} \exp\left(2\int_{r_{min}}^r\frac{\beta(s)}{s}\text{d}s\right)\text{d}r.
\end{equation*}
\begin{equation*}
    \bar{v_r^4}(R) = \frac{3}{\nu(R)}\exp\left(-2\int_{r_{min}}^{R}\frac{\beta(s)}{s}\text{d}s\right)\cdot\qquad
\end{equation*}
\begin{equation*}
    \qquad\int_R^\infty \frac{GM(r)\nu\sigma_r^2}{r^2} \exp\left(2\int_{r_{min}}^r\frac{\beta(s)}{s}\text{d}s\right)\text{d}r.
\end{equation*}

For distant spherical systems, only the projected velocity dispersion
$\siglos$ and the fourth order moment $\bar{v_{\rm los}^4}$ can be
measured, which in our case is given by

\begin{equation}\label{eq:LOS}
    \siglos^2(R) = \frac{2}{\Sigma(R)}\int_R^\infty \left(1-\beta\frac{R^2}{r^2}\right) \frac{\nu(r)\sigma_r^2(r) r}{\sqrt{r^2-R^2}}\text{d}r,
\end{equation}
\begin{equation}\label{eq:kappa}
    \bar{v_{\rm LOS}^4} = \frac{2}{\Sigma(R)}\int_R^\infty\frac{\nu \bar{v_r^4}r}{\sqrt{r^2-R^2}}g(r,R,\beta)\text{d}r
\end{equation}
\begin{equation*}
    g(r,R,\beta) = 1-2\beta\frac{R^2}{r^2}+\frac{\beta(1+\beta)}{2}\frac{R^4}{r^4},
\end{equation*}

where $\Sigma(R)$ denotes the surface mass density at radius $R$. As
done in \cite{Lokas+2005}, we will compare the kurtosis

\begin{equation*}
    \kappa_{\rm LOS}=\frac{\bar{v_{\rm LOS}^4}}{\sigma_{\rm LOS}^4}
\end{equation*}

between data and our model.

In the following, we present a non-parametric method for the solution
of equation \ref{eq:LOS} for the total gravitating mass density
$\rho(r)$, given tracer density profiles $\nu(r)$ and line-of-sight
velocity dispersion $\siglos(R)$. Following \cite{Jardel+2012} we
write the overall density profile $\rho(r)$ as

\begin{equation}
    \rho(r) = \rho_{\rm{DM}}(r)+\sum_{i=1}^{N_{\rm pop}}\left(\frac{M_*}{L}\right)_{i}\cdot \nu_i(r)
\end{equation}

and assume mass-to-light ratios $(M_*/L)_i$ to be constant with radius
for the tracer population $i$ in our mock datasets. We are ultimately
interested in $\rho_{\rm{DM}}$, which is
$\rho_{\rm{DM}}(r)\approx\rho(r)$ when neglecting the mass of the
tracer populations. This assumption is valid for our mock data and any
observed system with high dark matter content in the center, where the
tracer populations reside. We will drop said assumption when working
on real data.

We get the enclosed mass $M(<r)$ from the density via

\begin{equation}
    M(<r) = \int_0^r \rho(r) r^2 \text{d}r,
\end{equation}

which shows up in eq. \ref{eq:main}. In principle, the method can be
generalized to investigate alternative gravity models, if the
acceleration $GM(r)/r^2$ is replaced with the respective form of
$-\partial\Phi/\partial r$ in equation \ref{eq:Jeans}.

The degeneracy between mass $M$ and velocity anisotropy $\beta$ is
accounted for: For any non-isothermal system, we let the anisotropy
$\beta(r)$ vary as well. We checked that in the case of a simple
Hernquist profile, $\beta(r)\approx0$ is retrieved correctly.

\subsection{Representation in Radial Bins}
We describe now how we sample the space $[\rho, \nu_i, \beta_i]$
for distinct populations $i=1...N_{\rm pop}$ of stellar or gaseous
tracers. The main idea is to use a representation of the radial
profiles of each of those in $N_{\rm bin}$ independent bins,

\begin{equation}\label{eq:binning}
    \rho(r_j < r < r_{j+1}) = \rho_j, \qquad \forall j\in[1, N_{\rm bin}],
\end{equation}

and analogous for $\nu_i$.

The densities $\rho$ and $\nu_i$ are represented in terms of the
logarithmic density slopes $n(r_j)=-{\rm d}\log\rho(r)/{\rm d}\log
r|{r=r_j}$, $1\leq j\leq N_{bin}+3$, via

\begin{equation*}
    \rho(r) = \rho(r_{1/2})\cdot\exp\left[\int_{\log r_{1/2}}^{\log r}n(s)\text{d}s\right],
\end{equation*}

with the density at half-light radius $\rho(r_{1/2})=\rho_{1/2}$, and
$n(r)$ interpolated linearly in between bin radii
$r_{j-1}<r<r_{j}$. We prescribe three fudge $n(r_j)$ for $j\in\{N_{\rm
  bin}+1, N_{\rm bin}+2, N_{\rm bin}+3\}$ outside of the range where
data is given to enable sensible extrapolations towards high radii,
and two additional slopes $n_0 < 3, n_\infty>3$ for the asymptotic
density slopes towards $r=0$ and $r=\infty$, which are reached at half
the smallest and $r_\infty=10$ the largest bin radius.

The velocity anisotropy $\beta$ is allowed to vary freely in the
interval $[0, \infty[$ by sampling the modified, symmetric $\beta_*$
(\TODO{cite Read}),

\begin{equation*}
    \beta_* = \frac{\sigma_r^2-\sigma_t^2}{\sigma_r^2+\sigma_t^2} = \frac{\beta}{2-\beta} \in [-1,1]
\end{equation*}

with a polynomial

\begin{equation*}
    \beta_*(r) = \sum_{j=0}^{N_\beta} \beta_j\cdot\left(\frac{r_j}{r_{\text{max}}}\right)^j,
\end{equation*}

where $\beta_j\in [0,\beta_{\rm max}]$. With this method, unphysical
$\beta>1$ must be prevented with the correction

\begin{equation*}
    \beta_*(r_j) := \min(\max(\beta_*(r_j), -1),1).
\end{equation*}

This approach allows us to change the number of parameters $N_\beta$
easily, and to sample many qualitatively different models --
isotropic, radially biased, tangentially biased, and any gentle
transitions between those -- with very few parameters.

In a next step, $\siglosi(r)$ is calculated from $\rho(r)$,
$\nu_i(r)$, and $\beta_i(r)$ according eq. \ref{eq:LOS}. This is done
numerically, involving three integrations, which are performed with
polynomial extrapolations of the integrands up to infinity,
s.t. missing contributions from $r>r_{max}$ do not lead to an
artificial falloff of $\siglos$.

The last step involves comparison of the projected surface density
$\Sigma_i(r)$ -- calculated from the 3D tracer density $\nu_i(r)$ --
as well as the LOS velocity dispersion $\sigma_i(r)$, and the fourth
order moment of the velocity, if wished, to the respective 2D data
profiles for the tracer populations to get a likelihood based on the
overall goodness of fit

\begin{eqnarray}
    \chi^2 &=& \sum_{i=1}^N \chi_{\Sigma,i}^2 + \chi_{\sigma,i}^2 + \chi_{\kappa,i}^2,\\
    \chi_{\Sigma,i}^2 &=& \sum_{j=1}^{N_{\rm bin}} \left(\frac{\Sigma_{\text{data},i}(r_j)-\Sigma_{\text{model},i}(r_j)}{\varepsilon_\Sigma(r_j)}\right)^2,
\end{eqnarray}

with error $\varepsilon_\Sigma(r_j)$ on the data
$\Sigma_{\text{data},i}(r_j)$. Analogous expressions hold for
$\chi_{\sigma,i}^2$, and $\chi_{\kappa,i}^2$ if used. In absence of a
measured $\beta_i(r)$, we set $\chi_{\beta,i}^2=0$.

\subsection{Non-Parametric Representation}

There are a large number of parameters from the representation of the profiles,

\begin{equation}
    N_{\rm dim} = N_{\rm bin} + N_{\rm pop}\cdot(N_{\rm bin}+N_{\rm beta})
\end{equation}

with only very few constraints from physical priors. The functional
form of the profiles is not predescribed. These conditions is what we
call {\it non-parametric representation}.

\subsection{Parameter Space Sampling}
Early approaches to sampling the parameter space were performed with a
custom-built MCMC method. It proved unfeasible to sample the whole
parameter space on human timescales due to its high dimensionality.

To circumvent many likelihood evaluations around a local minimum, we
changed the underlying sampling method to use several MCMC walkers,
along the lines recently described in \citep{Nelson+2013} for their
parallel code RUN DMC to analyze radial velocity observations of
planetary systems.

A useful framework was found in MultiNest (\cite{Feroz+2009}), which
is a Bayesian nested sampling algorithm to generate posterior samples
from non-trivial distributions in high dimensions.

MultiNest samples the $n$-dimensional hypercube $\kappa=[0,1]^{N_{\rm  dim}}$, which needs to be translated into physical prior
distributions for each of the parameter profiles.

\subsection{Priors}
Following priors are included in the model, and help to constrain the parameter space to the physically possible range from the start:

\begin{enumerate}
    \item[1)] $0\leq \rho_j \leq 6$
    \item[2)] $0\leq \nu_j  \leq 6$
    \item[3)] $0\leq \beta^*_j\leq 3$
\end{enumerate}

Additional priors were first included, but then removed later for the
following reasons:

\begin{enumerate}
    \item[1)] A baryon prior $\rho(r) \geq
    \rho_b(r)-\epsilon_{\rho,b}(r) \forall r\geq0$, to ensure that no
    models with overall densities below the measured baryon density
    are considered any further, can only be used with mock data, where
    the baryonic density is known to be the sum of the individual
    tracer densities with known and constant mass to light
    ratios. This is not the case with real data, where biases towards
    higher luminosity stars or unaccounted-for populations and gas are
    missing from the analysis.

    \item[2)] A mass restriction $M(r>r_{max}) \leq M(<r_{max})/3.$,
    to reject any model which has more than $M_\infty/3$ in the
    extrapolated bins was introduced to keep the high radius density
    low, and exclude diverging mass integrals. A more concise approach
    of requesting $n(r>r_\infty)>n_{\rm thresh} \geq 3$ was chosen,
    which ensures finite mass toward $r\to\infty$.

    \item[3)] $\beta_i(r+\Delta r)-\beta_i(r) < 0.5$ to prevent any
    sudden jumps in $\beta_i$ is implicitly inherent in the functional
    setup of $\beta$ now, when higher order terms are excluded.
\end{enumerate}
