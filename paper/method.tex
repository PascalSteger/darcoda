\section{Method}\label{sec:method}

The collisionless Boltzmann equation for a spherical system with
gravitational potential $\Phi$,

\begin{equation}
\frac{\text{d}f}{\text{d}t} = \frac{\partial f}{\partial t} + \nabla_{\vec{x}} f\cdot\vec{v} - \nabla_{\vec{v}} f\cdot\nabla_{\vec{x}}\Phi = 0,
\end{equation}

describes the motion of tracer stars with distribution function
$f(\vec{x},\vec{v}\,)$.


In spherical coordinates $(r, \theta, \phi)$, the collisionless
Boltzmann equation then reads as

\begin{equation}
\frac{\partial f}{\partial t} + \dot{r}\frac{\partial f}{\partial r} + \dot{\theta}\frac{\partial f}{\partial \theta} + \dot{\phi}\frac{\partial f}{\partial \phi} + \dot{v}_r\frac{\partial f}{\partial v_r}+\dot{v}_\theta\frac{\partial f}{\partial v_\theta} +\dot{v}_\phi\frac{\partial f}{\partial v_\phi} = 0
\end{equation}

with velocities

\begin{eqnarray}
\dot{r}       &=& v_r,\\
\dot{\theta}  &=& v_\theta/r\\
\dot{\phi}    &=& v_\phi / r \sin\theta.
\end{eqnarray}

The assumption of steady state hydrodynamic equilibrium gives
$\partial f/\partial t=0$ and $\bar{v}_r=0$, and using spherical
symmetry $\bar{v}_\theta=0$, $\bar{v}_\phi=0$, with a unique
tangential velocity dispersion
$\sigma_\phi^2=\sigma_\theta^2=\sigma_t^2$ and a fourth order moment $\bar{v_r^4}$ we get

\begin{eqnarray}\label{eq:Jeans}
\frac{1}{\nu}\frac{\partial}{\partial r}(\nu\sigma_{r}^2) + 2\frac{\sigma_{r}^2-\sigma_{t}^2}{r} &=& -\frac{\partial \Phi}{\partial r} = -\frac{GM(<r)}{r^2},\\
\frac{\partial}{\partial r}(\nu\bar{v_r^4})+\frac{2\beta}{r}\nu\bar{v_r^4}+3\nu\sigma_r^2\frac{\partial\Phi}{\partial r}&=&0,
\end{eqnarray}

with enclosed mass $M(<r)$, gravitational constant $G =
6.67398\cdot10^{-11} \text{m}^3/\text{kg}\,\text{s}^2$. The departure
from spherical hydrostatic equilibrium $\sigma_r^2=\sigma_t^2$ is
measured by the anisotropy parameter

\begin{equation}
\beta \equiv 1-\frac{\sigma_t^2}{\sigma_r^2}
\end{equation}

with values in the range from $-\infty$ (purely circular orbits)
through $0$ (hydrostatic equilibrium) to $1$ (purely radial orbits).

Integrating both sides of equation \ref{eq:Jeans} gives the main
equation of this paper,

\begin{equation}\label{eq:main}
\sigma_r^2(R) = \frac{1}{\nu(R)}\exp\left(-2\int_{r_{min}}^{R}\frac{\beta(s)}{s}\text{d}s\right)\cdot\qquad
\end{equation}
\begin{equation*}
\qquad\int_R^\infty \frac{GM(r)\nu(r)}{r^2} \exp\left(2\int_{r_{min}}^r\frac{\beta(s)}{s}\text{d}s\right)\text{d}r.
\end{equation*}
\begin{equation*}
\bar{v_r^4}(R) = \frac{3}{\nu(R)}\exp\left(-2\int_{r_{min}}^{R}\frac{\beta(s)}{s}\text{d}s\right)\cdot\qquad
\end{equation*}
\begin{equation*}
\qquad\int_R^\infty \frac{GM(r)\nu\sigma_r^2}{r^2} \exp\left(2\int_{r_{min}}^r\frac{\beta(s)}{s}\text{d}s\right)\text{d}r.
\end{equation*}

For distant spherical systems, only the projected velocity dispersion
$\siglos$ and the fourth order moment $\bar{v_{\rm los}^4}$ can be
measured, which in our case is given by

\begin{equation}\label{eq:LOS}
\siglos^2(R) = \frac{2}{\Sigma(R)}\int_R^\infty \left(1-\beta\frac{R^2}{r^2}\right) \frac{\nu(r)\sigma_r^2(r) r}{\sqrt{r^2-R^2}}\text{d}r,
\end{equation}
\begin{equation}\label{eq:kappa}
\bar{v_{\rm LOS}^4} = \frac{2}{\Sigma(R)}\int_R^\infty\frac{\nu \bar{v_r^4}r}{\sqrt{r^2-R^2}}g(r,R,\beta)\text{d}r
\end{equation}
\begin{equation*}
g(r,R,\beta) = 1-2\beta\frac{R^2}{r^2}+\frac{\beta(1+\beta)}{2}\frac{R^4}{r^4}
\end{equation*}


where $\Sigma(R)$ denotes the surface mass density at radius $R$. 

As in \cite{Lokas+2005}, we will compare the kurtosis

\begin{equation*}
\kappa_{\rm LOS}=\frac{\bar{v_{\rm LOS}^4}}{\sigma_{\rm LOS}^4}
\end{equation*}

between data and our model.

In the following, we present a non-parametric method for the solution
of equation \ref{eq:LOS} for the total gravitating mass density
$\rho(r)$, given observed tracer density profiles $\nu(r)$ and
$\siglos(r)$, where $r$ denotes the projected two-dimensional radius
from the center of mass of the spherical system. Following
\cite{Jardel+2012} we write the overall density profile $\rho(r)$ as

\begin{equation}
    \rho(r) = \frac{M_*}{L}\cdot \nu(r)+\rho_{\rm{DM}}(r)
\end{equation}

and assume constant mass-to-light ratios $M_*/L$ for the tracer
populations in our mock datasets. We are ultimately interested in
$\rho_{\rm{DM}}$, which is $\rho_{\rm{DM}}(r)\approx\rho(r)$ when
neglecting the mass of the tracer populations. This assumption is
valid for our mock data and any observed system with high dark matter
content in the center, where the tracer populations reside. We will
drop said assumption when working on real data.

We get the enclosed mass $M(<r)$ from the density via

\begin{equation}
M(<r) = \int_0^r \rho(r) r^2 \text{d}r,
\end{equation}

which shows up in eq. \ref{eq:main}. In principle, the method can be
generalized to investigate alternative gravity models, if the
acceleration $GM(r)/r^2$ is replaced with the respective form of
$-\partial\Phi/\partial r$.

The degeneracy between mass $M$ and velocity anisotropy $\beta$ is
accounted for: For any non-isothermal system, we let vary the
anisotropy $\beta(r)$ as well. We checked that in the case of a simple
Hernquist profile, $\beta(r)\approx0$ is retrieved correctly.

We sample the parameter space $[\nu_i, \siglosi, \rho]$ for distinct
populations $i=1...N$ of stellar or gaseous tracers with a Monte Carlo Markov Chain method.

Early approaches were performed with a custom MCMC method. It proved
unfeasible to sample the whole parameter space on human timescales due
to its high dimensionality.

To circumvent many likelihood evaluations around a local minimum, we
changed the underlying sampling method to use several MCMC walkers,
along the lines recently described in \citep{Nelson+2013} for their
parallel code RUN DMC to analyze radial velocity observations of
planetary systems.

A useful framework was found in MultiNest (\cite{Feroz+2009}), which
is a Bayesian nested sampling algorithm to generate posterior samples
from non-trivial distributions in high dimensions.

There are a large number of parameters from the representation of the
radial profiles of each of those in $N_{bin}$ bins, with only very few
constraints from physical priors. The functional form of the profiles
is not predescribed. This is what we call {\it non-parametric}.

MultiNest samples the $n$-dimensional hypercube $\kappa=[0,1]^n$,
which needs to be translated into physical prior distributions for
each of the parameter profiles:

The overall density $\rho$ is represented in terms of the logarithmic
density slopes $n(r_i)=\kappa_{\rho_i} = (\kappa_{\rho_i})^2$, $1\leq
i\leq N_\rho+3$ and calculated as

\begin{equation*}
    \rho(r) = \rho(r_{1/2})\cdot\exp\left[\int_{r_{1/2}}^rn(s)\text{d}s\right],
\end{equation*}

with the density at half-light radius $\rho(r_{1/2})=$, and $n(r)$
interpolated linearly in between bin radii $r_{i-1}<r<r_{i}$. We
prescribe two additional slopes $n_0 < -3, n_\infty>-3$ for the
asymptotic density slopes towards $r=0$ and $r=\infty$, which are
reached at half the smallest and twice the largest bin radius.

The velocity anisotropy $\beta$ is allowed to vary freely in the
interval $[0, \infty[$ by sampling the modified, symmetric $\beta_*$
(\TODO{cite Read}),

\begin{equation*}
    \beta_* = \frac{\sigma_r^2-\sigma_t^2}{\sigma_r^2+\sigma_t^2} = \frac{\beta}{2-\beta} \in [-1,1]
\end{equation*}

with a polynomial s.t.

\begin{equation*}
    \beta_*(r) = \sum_i=0^{N_\beta} \kappa_{\beta_i}\cdot\left(\frac{r_i}{r_{\text{max}}}\right)^i,
\end{equation*}

where $\kappa_{\beta_i}\in [0,1]$. With this method, unphysical
$\beta>1$ are possible, which we prevent with a correction

\begin{equation*}
    \beta_*(r_i) := \min(\max(\beta_*(r_i), -1),1).
\end{equation*}

This approach allows us to change the number of parameters $N_\beta$
easily, and to sample many qualitatively different models --
isotropic, radially biased, tangentially biased, and any gentle
transitions between those -- with very few parameters.

In a next step, $\siglosi(r)$ is calculated from $\nu(r)$, $\rho(r)$,
and $\beta(r)$ according eq. \ref{eq:LOS}. This is done numerically,
involving three integrations, which are performed with polynomial
extrapolations of the integrands up to infinity, s.t. missing
contributions from $r>r_{max}$ do not lead to an artificial falloff of
$\siglos$. The additional parameters of the slopes are calculated from
the values at lower radii, thus preventing the introduction of any
further parameters.

The last step involves comparison of the projected $\nu_i(r)$,
$\sigma_i(r)$ and $\beta_i(r)$, if available, to the data for the
tracer populations to get a likelihood based on the overall
goodness of fit

\begin{eqnarray}
\chi^2 &=& \sum_{i=1}^N \chi_{\nu,i}^2 + \chi_{\sigma,i}^2 + \chi_{\beta,i}^2\\
\chi_{\nu,i}^2 &=& \sum_{j=1}^{N_{bin}} \left(\frac{\nu_{i,\text{data}}(r_j)-\nu_{i,\text{model}}(r_j)}{\epsilon_\nu(r_j)}\right)^2.
\end{eqnarray}

and accordingly for $\chi_{\sigma,i}^2$ and $\chi_{\beta,i}^2$. In
absence of a measured $\beta_i(r)$, we set $\chi_{\beta,i}^2=0$.

\subsection{Binning characteristics}

The number of bins for $\rho, \nu_i, \beta_i$, and $\sigma_i$ are free
parameters. They are set to fulfill

\begin{equation}
n_\nu = n_\sigma = n_\beta = n_\rho \leq n_{\text{data}}
\end{equation}

with number of datapoints $n_{\text{data}}$. This choice simplifies
integration greatly, and prevents invention of information on scales
smaller than the frequency of datapoints.

\TODO{check that nbin is set s.t.}

\begin{equation}
\chi^2_{red} = \frac{\chi^2}{n_{\text{data}} - n_\nu - n_\sigma - n_\beta - n_\rho -1}
\end{equation}

is minimized, and still the whole parameter space is tracked.

The whole parameter space for $\beta_i$ is sampled with $n_\beta=12$
if $n_{\text{data}}=30$ in the case of 10000 tracers.

The dark matter density is calculated by subtracting the measured
baryon density from the dynamical mass density.

\subsection{Priors}
Following priors are included in the model, and help to reject
unphysical samplings from the start:

\begin{enumerate}
    \item[1)] baryon prior: $\rho(r) \geq
    \rho_b(r)-\epsilon_{\rho,b}(r) \forall r\geq0$, ensures that no
    models with overall densities below the measured baryon density
    are considered any further;

\item[2)] mass restriction: $M(r>r_{max}) \leq M(<r_{max})/3.$, rejects any
  model which has more than $M_\infty/3$ in the extrapolated bins;

\item[3)] $\beta_i(r+\Delta r)-\beta_i(r) < 0.5$: prevent any sudden
  jumps in $\beta_i$;
\end{enumerate}

We show in the appendix what effects the disabling of these priors
have.
