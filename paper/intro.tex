
\section{Introduction}\label{sec:introduction}

% dark matter in CDM
% cusp-core problem

Cosmological $\Lambda$CDM simulations of representative patches of the
Universe predict the dark matter to assemble self-similarly. Assuming
that only dark matter might influence the physics on all scales down
to the stellar regime, \citet{Navarro+1997} found that the density
profiles of the resulting halos are best described by a function
diverging as $r^{-1}$ towards the center, turning over to $r^{-2}$ at
a scale radius, and falling off as $r^{-3}$ towards infinity. The
asymptotic value $\alpha = d \log\rho/d \log r = -1$ in the center is
challenged by observations.

Observations of low surface brightness galaxies measure rotation
curves and deduce from them a constant density -- $\alpha=0$ -- below
a rather large scale radius $r_S$ \bcite{DeBlok+2001},
\bcite{McGaugh+2001}, contrary to the theoretical predictions. This
fact became known as the cusp/core problem.

% solutions to cusp-core: observations
Many solutions have been proposed. It was deemed probable that
difficulties in data modelling prevent observations from resolving
cusps. \TODO{citation}

% solutions to cusp-core: 
Another solution was to propose that dark matter is not cold as
assumed, but has a high streaming velocity with a warm component
smearing out density peaks below a certain scale.

Dark matter could also be self-interacting, as e.g. postulated by
(\citet{SpergelSteinhardt2000}, \citet{Vogelsberger+2012}), and have a
large scattering cross-section and low annihilation or dissipation
cross-section, and thus prevent formation of overly-dense cusps.

% solutions to cusp-core: theory: stellar feedback, higher density
% threshold, individual starforming regions

For the $n$-body simulations in question, it became clear that
baryonic physics plays a crucial role for the buildup of the overall
density profile on small scales, and thus dark matter.

In particular, modelling of stellar feedback, the introduction of a
higher density threshold for star formation and an increase of
resolution for treatment of individual starforming regions in a
cosmological context led simulations to reveal dwarf galaxies with
shallow central density slopes in dark matter
\bcite{Governato+2010}. This compares well with what is found in
THINGS dwarf galaxies \bcite{Oh+2011}.

% comparison to local dwarf galaxies
A possible way for clarification in this issue are dwarf spheroidal
galaxies, lying relatively close to the Sun such that they can be
resolved, often allowing us to observe individual stars. We need a
method to model their mass distribution from the observations to
investigate whether they exhibit a cusp or a core, and whether
simulations are able to reproduce the dark matter distribution.

% mass modelling in general: schwarzschild and jeans
An early approach for mass modelling general triaxial systems in
dynamical equilibrium was proposed by \citet{Schwarzschild1979}: Based
on a postulated density profile and a corresponding gravitational
potential, an ensemble of orbits are calculated and superposed to
yield the underlying density profile.


% motion of globular clusters in dwarfs
Another mass modelling approach is to use the motion of globular
clusters inside dwarf galaxies \bcite{Goerdt+2006},
\bcite{Cole+2012}. If the density distribution follows a cored
profile, globular clusters will not fall in, or will even get pushed
out of the core if they formed inside. In Fornax dSph, there is
evidence for a core \bcite{Read+2006}.


% \citet{Breddels+2012} cannot distinguish
% between cuspy or cored profiles for Sculptor.
Yet another method makes use of the Jeans equations encompassing tracer
density, velocity dispersion and the gravitational potential to solve
for the potential, and ultimately get the underlying dark matter
density \bcite{BinneyTremaine2008}. Mostly, a functional form with
some free parameter(s) is assumed for the density profile, and fitting
routines are used to yield the best agreeing form. While this can give
good agreement on specific systems (\TODO{cite King}), it only ever
allows the particular class of models to be fitted. A more elegant way
would be to infer the functional form as well from the data, in a
non-parametric way.


% motivation for non-parametric mass-modelling
In this paper, we propose a new non-parametric mass-modelling
technique based on the Jeans approach, with no assumptions on the
functional form of the dark matter density, nor the velocity
anisotropy profile. We sample all of these profiles in $N_{bin}$ bins,
and deduce the most likely profiles.


% positive points: why to use non-parametric methods
Our method has following advantages over other methods presented so far:
\begin{enumerate}
\item no assumptions on the functional form of the underlying dark
  matter;
\item applicable to any gravitational model, since Jeans equation and
  Poisson equation are solved each on their own;
\item robust to noise in the data, as no numerical differentiation is
  used as soon as the three-dimensional density model is set up.
\end{enumerate}

% what about: non-sphericity, ellipticity? later on
We cannot employ our method on all systems, though, as the strong
assumption of spherical symmetry enters in a first step. Beware that
dSph galaxies of the local group are known to be slightly
non-spherical, with an average ellipticity of 0.3 \bcite{Mateo1998}.

We find at this early stage, as well, that we might be susceptible to
the mass/velocity anisotropy, due to the following reason.

% mass beta anisotropy
An assumption on the velocity anisotropy profile is required to get a
mass density in the case of lowest-order Jeans equations. Since it is
not known a priori and hard to measure observationally how much the
system is supported by rotational motion, this leads to a degeneracy
between mass and velocity anisotropy.

\TODO{put the following three paragraphs in Discussion?}

% Walker/Penarrubia 2011: 2 tracers
\citet{WalkerPenarrubia2011} split the stellar tracers of Fornax dSph
into two populations with different metallicities and half-mass radii,
and make use of the fact that the enclosed mass at the half-mass
radius is independent of the underlying velocity anisotropy profile,
to get two points in the center of the dwarf galaxy, from which
constraints on the inner DM density slope can be drawn. One further
line of investigation will show whether splitting into $N$ populations
might yield a better result with our method.

% break via higher orders?
Higher order Jeans equations could help to break the degeneracy as
well \bcite{Lokas2002}. As pointed out by \bcite{Richardson+2013},
this might not be enough, as any additional order introduced more
anisotropy parameters than they help to constrain. Introducing virial
shape parameters yields a much better handle on the degeneracy.

Another degeneracy between inner DM slope and concentration shows up
as well. With better data available for the local dSphs, distribution
function based methods may be used.
