
\section{Introduction}\label{sec:introduction}

% dark matter in CDM
% cusp-core problem

Cosmological $\Lambda$CDM simulations of representative patches of the
Universe predict the dark matter to assemble self-similarly.

Assuming that only dark matter might influence the physics on all
scales down to the stellar regime, \citet{Navarro+1997} found that the
density profiles of the resulting halos are best described by a
function diverging as $r^{-1}$ towards the center. Given that only a
finite amount of dark matter is available in the halo, there exists a
very small lower bound for a turnover radius, where the density
approaches a constant value.

Observations of low surface brightness galaxies measure rotation
curves and deduce from them a constant density below a rather large
scale radius $r_S$ \bcite{DeBlok+2001}, \bcite{McGaugh+2001}, contrary to
the theoretical predictions. This fact became known as the cusp/core problem.


% solutions to cusp-core: observations

Many solutions have been proposed to solve it. It was deemed possible
that difficulties in data modelling prevent observations from
resolving cusps.

% solutions to cusp-core: 

From a theoretical point, it was proposed that dark matter is not cold
as assumed, with a warm component smearing out density peaks below a
certain scale.

Dark matter could also be self-interacting, as e.g. postulated by
(\citet{SpergelSteinhardt2000}, \citet{Vogelsberger+2012}), and have a
large scattering cross-section and low annihilation or dissipation
cross-section, and thus prevent formation of overly-dense cusps.

% solutions to cusp-core: theory: stellar feedback, higher density
% threshold, individual starforming regions

For the simulations in question, it became clear that baryonic physics
plays a crucial role for the buildup of the overall density profile on
small scales, and thus dark matter.

In particular, modelling of stellar feedback, the introduction of a
higher density threshold for star formation and an increase of
resolution for treatment of individual starforming regions in a
cosmological context led simulations to reveal dwarf galaxies with
shallow central density slopes in dark matter
\bcite{Governato+2010}. This compares well with what is found in
THINGS dwarf galaxies \bcite{Oh+2011}.



% comparison to local dwarf galaxies

Another type of dwarf galaxies, the dwarf spheroidals, lie closer to
the Sun and can be resolved much better, often allowing us to observe
individual stars. We need a method to model their mass distribution
from the observations to investigate whether they exhibit a cusp or
a core, and whether simulations are able to reproduce the dark matter
distribution.


% mass modelling in general: schwarzschild and jeans

An early approach for general triaxial systems in dynamical
equilibrium was proposed by \citet{Schwarzschild1979}: Based on a
density profile and a corresponding gravitational potential, an
ensemble of orbits are calculated and superposed to yield the
underlying density profile.

% \citet{Breddels+2012} cannot distinguish
% between cuspy or cored profiles for Sculptor.

Another method makes use of the Jeans equations encompassing tracer
density, velocity dispersion and the gravitational potential to solve
for the potential, and ultimately get the underlying dark matter
density \bcite{BinneyTremaine2008}. Mostly, a functional form with
some free parameters is assumed for the density profile, and fitting
routines are used to yield the best agreeing form.

% mass beta anisotropy

Another assumption is required for the velocity anisotropy profile to
get a mass density in the case of lowest-order Jeans equations. Since
it is not known a priori and hard to measure observationally how much
the system is supported by rotational motion, this leads to a
degeneracy between mass and velocity anisotropy.

% break via higher orders

Higher order Jeans equations can help to break this degeneracy
\bcite{Lokas2002}. Another degeneracy between inner DM slope and
concentration shows up, though. With better data available for the
local dSphs, distribution function based methods may be used.


% motion of globular clusters in dwarfs

Yet another approach is to use the motion of globular clusters inside
dwarf galaxies \bcite{Goerdt+2006}, \bcite{Cole+2012}. If the density
distribution follows a cored profile, globular clusters will not fall
in, or will even get pushed out of the core if they formed inside. In
Fornax dSph, there is evidence for a core \bcite{Read+2006}.


% Walker/Penarrubia 2011: 2 tracers

\citet{WalkerPenarrubia2011} split the stellar tracers into two
populations with different metallicities and half-mass radii, and make
use of the fact that the enclosed mass at the half-mass radius is
independent of the underlying velocity anisotropy profile, to get two
points in the center of the dwarf galaxy, from which constraints on
the inner DM density slope can be drawn.




% motivation for non-parametric mass-modelling
We want to get the full density profile, though.

In this paper, we propose a new non-parametric mass-modelling
technique based on the Jeans approach, with no assumptions on the
functional form of the dark matter density, nor the velocity
anisotropy profile.


% what about: non-sphericity, ellipticity? later on

We assume spherical symmetry in a first step. Beware that dSph
galaxies of the local group are known to be slightly non-spherical,
with an average ellipticity of 0.3 \bcite{Mateo1998}.


% positive points: why to use non-parametric methods

Our method has following advantages over other methods presented so far:

\begin{enumerate}
\item no assumptions on the functional form of the underlying dark
  matter;
\item applicable to any gravitational model, since Jeans equation and
  Poisson equation are solved each on their own;
\item robust to noise in the data, as no numerical differentiation is
  used as soon as the three-dimensional density model is set up.
\end{enumerate}
