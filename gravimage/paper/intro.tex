\section{Introduction}\label{sec:introduction}

% dark matter in CDM
% cusp-core problem

Cosmological $\Lambda$CDM simulations predict a hierarchical
self-similar assembly of dark matter halos
\citep[e.g.][]{WhiteRees1978, NavarroFrenkWhite1996}. Modelling only
the dark matter fluid in the absence of `baryons' (stars and gas),
\citet{DubinskiCarlberg1991} found that the density profile of
resulting halos are best described by a split power law that diverges
as $r^{-1}$ at the centre. \citet{NavarroFrenkWhite1996} demonstrated
that this profile is universal, giving a good match to
halos of all mass from those hosting dwarf galaxies to those hosting
galaxy clusters; they suggested a fitting function:

\begin{equation}
    \rho(r) = \rho_0 \left(\frac{r}{r_0}\right)^{-1}\left(1 + \frac{r}{r_0}\right)^{-2}
    \label{eqn:nfw}
\end{equation}
that has become known as the `NFW' profile.

While $\Lambda$CDM has performed remarkably well on large scales
\citep[e.g.][]{2002PhRvD..66j3508T}, the above prediction has long been at
tension with observational data from dwarf and low surface brightness (LSB)
galaxies. \citet{FloresPrimack1994} and \citet{Moore1994} were the first to show
that fits to dwarf galaxy rotation curves favour a dark matter density profile
with a central constant density core: $\left. d\ln \rho / d\ln r\right|_{r=0}
\equiv \alpha = 0$, rather than a cusp. Since then, similar results have been
reported for a wide range of gas rich dwarf and low surface brightness (LSB)
galaxies \citep[e.g.][]{DeBlok+2001, McGaugh+2001, DeBlok+2008,
  HagueWilkinson2014}; a result that is robust to known observational
uncertainties and model systematics
\citep[e.g.][]{KuzioDeNarayKaufmann2011}. This now long-standing discrepancy
between theory and observation has become known as the `cusp-core' problem
\citep[for a review see e.g.][]{DeBlok2010}.

One proposed solution to the cusp-core problem is to invoke stellar feedback,
not modelled in the pure dark matter simulations discussed
above. \citet{NavarroEkeFrenk1996} were the first to suggest that impulsive
winds driven by supernovae could cause the dark matter halo to expand erasing
the central cusp. However, \citet{GnedinZhao2002} demonstrated that, once the
angular momentum barrier to gas collapse is taken into account, the maximum
effect of a blow out is small. \citet{ReadGilmore2005} showed that this problem
can be overcome if star formation proceeds in multiple bursts, gradually
transforming a cusp to a core over several cycles of star formation. This
mechanism appears to be what is at play in recent high resolution simulations of
dwarf galaxies that model both the dark matter fluid and the baryons
\citep{MashchenkoWadsleyCouchman2008, Governato+2010, PontzenGovernato2012,
  Teyssier+2013}. (For an elegant analytic treatment of the effect see
\citealt{PontzenGovernato2012}; and for a review see
\citealt{PontzenGovernato2014}.) \citet{Teyssier+2013} point to two key
observational predictions of such a scenario: (i) star formation should be
bursty with a duty cycle of $\sim$ a dynamical time; and (ii) the stars should
be collisionlessly heated along with the dark matter, producing vertically hot
stellar discs even in isolated galaxies. Such predictions appear to be supported
by the latest data \citep{2012ApJ...750...33L,Kauffmann2014}.

If baryons really do transform cusps to cores then this is bad news for
constraining dark matter models. Exotic models that produce cores
\citep[e.g.][]{2013MNRAS.431L..20Z} would become indistinguishable from
`vanilla' cold dark matter. This motivates pushing to ever smaller scales where
the baryonic effects should diminish. \citet{Penarrubia+2012} have recently
suggested that supernova feedback will no longer provide enough energy for
cusp-core transformations below some critical stellar mass $M_{\rm c} \simlt
10^6$\,M$_\odot$, though the precise value of $M_{\rm c}$ remains under
investigation \citep[e.g.][]{2014ApJ...789L..17M}. The dwarf spheroidal galaxies
(dSphs) that orbit the Milky Way and Andromeda straddle this critical stellar
mass, making them prime targets for measuring their dark matter density
profiles. They also have the added advantage that most of their gravitating mass
is dark with typically negligible contributions from stars and/or gas. The most
massive, Fornax, has some $\sim 10^7$\,M$_\odot$ in stars
\citep{ColemanDeJong2008}, lying above $M_{\rm c}$; while the smallest, Segue 1
(if it is indeed a galaxy; \citealt{2009MNRAS.398.1771N,2011ApJ...738...55M}),
has just $\sim 1000$\,M$_\odot$ \citep{Belokurov+2007}, lying well below $M_{\rm
  c}$.

Due to their proximity to their host galaxies, dSphs differ from the
dwarf and LSB galaxies discussed above in that they are almost
completely devoid of gas \citep[e.g.][]{Gatto+2013}. This presents a
challenge because stellar orbits, unlike gas, can cross. If only
radial velocities are available, this leads to a strong degeneracy
between the dark matter density profile and the orbit distribution of
the stars, typically parameterised by the velocity anisotropy
\citep[e.g.][]{Wilkinson+2002}:

\begin{equation}
    \label{eqn:beta}
    \beta \equiv 1 - \frac{\sigma_\theta^2 + \sigma_\phi^2}{2\sigma_r^2}
\end{equation}
where $\sigma_{r,\theta,\phi}$ are the velocity dispersions in
spherical polar coordinate directions $r, \theta$ and $\phi$,
respectively. (An anisotropy $\beta = 0$ corresponds to isotropic
velocity dispersions, while $\beta = 1$ is pure radial and $\beta =
-\infty$ pure tangential.)

\citet{Battaglia+2008} suggested that the above degeneracy could be broken if
stars are split into multiple tracer populations by either metallicity or
abundance. They performed such an analysis on the Sculptor dSph, weakly
favouring a dark matter core. This analysis was further refined and applied to
four dSphs by \citet{WalkerPenarrubia2011} (hereafter WP11). They also favoured
cores but with higher statistical significance, supporting earlier timing
arguments for UMi and Fornax \citep{Kleyna+2003, Goerdt+2006, Read+2006,
  Cole+2012}. However, recently the literature has become divided on the issue
of cusps or cores in the dSphs. An analysis of four dwarfs using a
non-parametric Schwarzschild method \citep{Schwarzschild1979} favoured a wide
range of density profiles \citep{Jardel+2013}; simple distribution function
models \citep{AmoriscoEvans2012} or `virial parameter' models \citep{Evans+2011}
support the findings of WP11; while single component Schwarzschild
\citep{BreddelsHelmi2013} and higher order Jeans \citep{RichardsonFairbairn2013}
analyses conclude that the data are simply not sufficient to say one way or the
other \citep[see also the discussion in][]{2014arXiv1404.5958B}. Finally,
\citet{2014arXiv1406.6079S} have recently pointed out that some of the key
assumptions in the WP11 analysis may be violated if the density profile of one
or more of the tracers is cuspy, leading to central cores being artificially
favoured over
cusps.

% Each of the above analyses relies on often disjoint assumptions and
% methodologies, while only a few of the
% authors have presented detailed tests on mock data (e.g WP11). This makes it
% challenging to reconcile these
% different works and/or obtain a robust constraint on the mass profiles of
% nearby dSphs.

In light of the above discrepancies, in this paper we present a new
non-parametric Jeans modelling tool: \GravImage. Our goal is to assume only
equilibrium and spherical symmetry, allowing the model full freedom
otherwise. We refer to this approach as non-parametric, though really what we
mean is that we have a model with far more parameters than the available data
constraints. Since this means that we are necessarily under-constrained, we are
then forced to build model ensembles and explore parameter degeneracies. For
this, we use the efficient \MultiNest\ technique \citep{Feroz+2009}. We support
multiple tracers that can be simultaneously modelled in a single underlying
potential (each with its own free form velocity anisotropy); and we allow for
velocity moments up to fourth order to be constrained. Similar such approaches
have been attempted in the literature before. \citet{2011ApJ...738..186I}
(section 5.1) present a non-parametric spherical Jeans solver for modelling
globular clusters; \citet{JardelGebhardt2013} present a non-parametric
Schwarzschild method; and \citet{2001AJ....122..232C} present a non-parametric
method for modelling the black hole at the centre of our Galaxy. However, our
new method is more general than these previous
works. \citet{2011ApJ...738..186I} -- since they are modelling globular star
clusters -- assume that the mass distribution is known, fitting only for the
mass to light ratio and $\beta(r)$; \citet{JardelGebhardt2013} present a method
very different from that discussed here (and therefore complementary); however,
they do allow for multiple populations, or test their methodology on mock
data. Finally, \citet{2001AJ....122..232C} present an elegant non-parametric
distribution function method but it relies on assuming an isotropic velocity
distribution function ($\beta = 0$). This may be reasonable for stars orbiting a
supermassive black hole, but it likely a poor assumption when modelling
dSphs. Using spherical mock data, we set out to determine what type and quality
of data is sufficient to constrain the logarithmic cusp slope within dSphs. We
also use triaxial mock data to test what happens when our method is pushed
beyond its regime of validity. We will present applications of our method to
real data in forthcoming papers.

This paper is organised as follows. In \S\ref{sec:method} we introduce
the method. In \S\ref{sec:results} we test our method on
mock data. Finally, in \S\ref{sec:conclusions} we present our conclusions.

%%% Local Variables:
%%% mode: latex
%%% TeX-master: "Steger_2014_GravImage"
%%% End:
