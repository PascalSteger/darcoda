
\section{Introduction}\label{sec:introduction}
\subsection{Cusp vs Core}
Without baryonic physics, simulations of structure formation predict a
universal density profile of the NFW profile \cite{Moore1999} with an
asymptotic slope towards the center of $\alpha=1$.

Analysis of observations on the other hand tends to find more constant
densities in the innermost parts of small systems,
\cite{FloresPrimack1994}, \cite{Oh+2011}. This has become known as
cusp-core problem.

Baryonic outflows and sudden changes of the gravitational potential
were proposed as one possibility to create cores in simulations
(\cite{Read+2006}, \cite{Pontzen+2013}, \cite{Teyssier}). Another
effect would be dynamical stirring \cite{Goerdt+2006}.


\subsection{Jeans Analysis}
\TODO{Jeans analysis, mass-beta anisotropy breaking Richardson2014 and before}


\subsection{Populations of Tracer Stars}
\citet{Battaglia 2008} try to break the mass-anisotropy degeneracy by
splitting stars into multiple populations.

\citet{WalkerPenarrubia2011}




\citep{Tolstoy+2004} find distinct populations of stars in Sculptor,
\citep{Battaglia+2006}, \citep{Battaglia2011} for Fornax and Sextans,
\citep{Harbeck+2001} and \citep{Bellazzini+2001} for Carina, Tucana
and Andromeda satellites.

\subsection{Analyses with Parametric Methods}
Jeans analysis is normally performed with a parametrized form of the
density profile. A short overview of such methods as applied on Sculptor follows:

\citep{Battaglia+2008} introduce population splitting, and find a core
with low significance. \citep{AmoriscoEvans2012} use population
splitting and a distribution function approach, to find a core at high
significance.

%\citep{Breddels et al. 2011  Schwarzschild  cannot tell}
\citep{RichardsonFairbairn2014} with virial parameters on one
population find a cusp. \citep{Strigari+2014} use population
splitting and a distribution function appreach, finding a cusp.




\citep{WalkerPenarrubia2011} use population
splitting, and find a core for Fornax with high significance.
\TODO{more Fornax findings}


\citep{Ibata+2011} use a method which is not based on a parametrized profile \TODO{describe}.



\subsection{Non-Parametric Method}
In a previous paper \citet{Steger+2014} we showed a new non-parametric
method for Jeans-modelling of 1-dimensional systems.



\section{Data}
We infer the baryonic density profile for Fornax by data of
\cite{deBoer+2012b}, which measure photometry in B, V, and I filters
for 270.000 stars.


\cite{deBoer+2013} identify several substructures in the B-(V-B)
plane.  \TODO{Justin: We could use the color-magnitude diagram splits
  for stellar populations, instead of only using the Mg index. Thus,
  we would introduce a second dimension, and could even use the same
  cuts as in there. Needs a join on stellar positions from the
  deBoer to the Walker data.}

This represents our baryonic profile, which is subtracted from the
overall density profile to yield the dark matter profile.


The kinematic data used for our study is based on \citep{Walker2009b}
(see \citep{Walker2009a} for a detailed description of the
acquisition). It contains 2.500 measurements of the kinematics of member stars.

\TODO{Justin: Walker+2009b seems to determine Fe and Mg line widths independently}




%%% Local Variables:
%%% mode: latex
%%% TeX-master: "Steger+_2014_Fornax"
%%% End:
