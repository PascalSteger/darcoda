\documentclass[10pt]{report}
\begin{document}
%
\begin{titlepage}
 \title{Rapport 2009-03-16}
 \author{P.S.P. Steger}
 \maketitle
\end{titlepage}
%
%
\paragraph{R\"uckblick}
%
\begin{itemize}
\item Aufbau Verzeichnis relevanter Literatur, Zusammenfassung, Definitionen
\item Einleitung Paper
\item neue Version AMIGA, AHF korrigiert
\item Aufbereitung des alten Codes von O. Hahn begonnen: Virial Properties mit overdensity wieder aktiviert, zus\"atzliches Entfernen ungebundener Teilchen revidiert, OOP
\item Vorlesungen wieder begonnen
\end{itemize}
%
\paragraph{Ausblick}
%
\begin{itemize}
\item Weiterarbeit an (neuem?) Paper erst nach Tests des neuen Codes
\item Literatur r\"uckverfolgen
\item Definition Themengebiet, Titel der Masterarbeit; abzuschliessen bis sp\"atestens Ende FS2012
\end{itemize}
%
%
\paragraph{Pendenzen}
\begin{itemize}
\item checks for numerical effects, as proposed by O. Hahn
\item paper
\item Binney/Tremaine: chapter 4..8,10..
\item \"Anderungen f\"ur AMR?
\end{itemize}
%
\subparagraph{Diverses}
%
\begin{itemize}
\item picture of galaxy: extraction of dm halo mass, angular momentum, density distribution
\item correction of images for weak lensing: get 'real' dm distribution
\item distribution of dark nrj: uniform/follows nrj/concentrated in voids?
\item distribution of dark matter: structured/uniform: which scale for splitting point?
\item heating of gas in simulation, quantitative, as fct of density,z,
\item simulate mergers, looking at dm halo (mass loss, final angular momentum, conservation of momentum)
\item galaxies as spinning tops: precession of stars/gas/dm shells as fct of environment/mass/axes/angular momentum
\item how can disk survive tidal stresses from near encounters of dm subhalos?
\item influence of dm gravitational field on spectra of stars/galaxies?
\item fractal properties of mass distribution as fct of scale
\end{itemize}
%
\end{document}

