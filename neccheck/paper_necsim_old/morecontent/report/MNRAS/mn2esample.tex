% mn2esample.tex
%
% v2.1 released 22nd May 2002 (G. Hutton)
%
% The mnsample.tex file has been amended to highlight
% the proper use of LaTeX2e code with the class file
% and using natbib cross-referencing. These changes
% do not reflect the original paper by A. V. Raveendran.
%
% Previous versions of this sample document were
% compatible with the LaTeX 2.09 style file mn.sty
% v1.2 released 5th September 1994 (M. Reed)
% v1.1 released 18th July 1994
% v1.0 released 28th January 1994

\documentclass[useAMS,usenatbib]{mn2e}

% If your system does not have the AMS fonts version 2.0 installed, then
% remove the useAMS option.
%
% useAMS allows you to obtain upright Greek characters.
% e.g. \umu, \upi etc.  See the section on "Upright Greek characters" in
% this guide for further information.
%
% If you are using AMS 2.0 fonts, bold math letters/symbols are available
% at a larger range of sizes for NFSS release 1 and 2 (using \boldmath or
% preferably \bmath).
%
% The usenatbib command allows the use of Patrick Daly's natbib.sty for
% cross-referencing.
%
% If you wish to typeset the paper in Times font (if you do not have the
% PostScript Type 1 Computer Modern fonts you will need to do this to get
% smoother fonts in a PDF file) then uncomment the next line
% \usepackage{Times}

%%%%% AUTHORS - PLACE YOUR OWN MACROS HERE %%%%%


%%%%%%%%%%%%%%%%%%%%%%%%%%%%%%%%%%%%%%%%%%%%%%%%

\title[Dust envelopes around RV Tauri stars]{Dust envelopes around RV Tauri stars}
\author[A. V. Raveendran and A. N. Other]{A. V. Raveendran$^{1}$\thanks{E-mail:
email@address (AVR); otheremail@otheraddress (ANO)} and A. N.
Other$^{2}$\footnotemark[1]\thanks{This file has been amended to
highlight the proper use of \LaTeXe\ code with the class file.
These changes are for illustrative purposes and do not reflect the
original paper by A. V. Raveendran.}\\
$^{1}$Indian Institute of Astrophysics, Bangalore 560034, India\\
$^{2}$Building, Institute, Street Address, City, Code, Country}
\begin{document}

\date{Accepted 1988 December 15. Received 1988 December 14; in original form 1988 October 11}

\pagerange{\pageref{firstpage}--\pageref{lastpage}} \pubyear{2002}

\maketitle

\label{firstpage}

\begin{abstract}
In the {\it IRAS\/} [12]--[25],  [25]--[60] colour--colour diagram, RV
Tauri stars are found to populate cooler temperature regions
$(T<600\,\rmn{K})$, distinctly different from those occupied by  the
oxygen and carbon Miras. The {\it IRAS\/} fluxes are  consistent with the
dust density in the envelope varying as the inverse square of the
radial distance, implying that the grain formation processes in these
objects are most probably continuous and not sporadic. It is found that
the spectroscopic subgroups A and B are well separated in the
far-infrared two-colour diagram, with group B objects having
systematically cooler dust envelopes. We interpret this as being due to
a difference in the nature of grains, including the chemical
composition, in the two cases.
\end{abstract}

\begin{keywords}
circumstellar matter -- infrared: stars.
\end{keywords}

\section{Introduction}

It has been well established that RV Tauri variables  possess
infrared emission far in excess of their expected  blackbody
continuum, arising from their extended cool dust envelopes
\citep{b7,b5,b6}. Recently, \citet{b15} and \citet{b9} have given
detailed descriptions of the near-infrared properties of RV Tauri
stars. In this paper we present an analysis of the {\it IRAS\/}
data of RV Tauri stars with the help of the far-infrared
two-colour diagram and a grid computed using a simple model of the
dust envelope. Such two-colour plots have already been employed
extensively by several investigators to study the circumstellar
envelopes around oxygen-rich and carbon-rich objects which are in
the late stages of stellar evolution \citep{b10,b25,b23,b24}.

Table 1 summarizes the basic data on the 17 objects  detected at
\hbox{60\,$\umu$m}. Apart from the {\it IRAS\/} identification and
the flux densities at 12-, 25-, 60-  and 100-$\umu$m wavebands, it
gives the spectroscopic groups of \citet{b20}, the light-curve
classes of \citet{b13} and the periods of light variation. The
list, which contains  about 20 per cent of all the known RV Tauri
stars, is  essentially the same as that given by \citet{b12}. The
spectroscopic subgroups are from either \citet{b20} or
\citet{b15}.
\begin{table*}
 \centering
 \begin{minipage}{140mm}
  \caption{Data on the RV Tauri stars detected by {\it IRAS}.}
  \begin{tabular}{@{}llrrrrlrlr@{}}
  \hline
   Name     &            & \multicolumn{4}{c}{Flux density (Jy)%
  \footnote{Observed by {\em IRAS}.}}\\
   Variable & {\it IRAS} & 12$\,\umu$m & 25$\,\umu$m & 60$\,\umu$m
     & 100$\,\umu$m & Sp. & Period & Light- & $T_0\,(\rmn{K})$ \\
        &  &  &  &  &  & group & (d) & curve \\
        &  &  &  &  &  &       &     & type  \\
 \hline
 TW Cam & 04166$+$5719 & 8.27 & 5.62 & 1.82 & $<$1.73 & A & 85.6 & a & 555 \\
 RV Tau & 04440$+$2605 & 22.53 & 18.08 & 6.40 & 2.52 & A & 78.9 & b & 460 \\
 DY Ori & 06034$+$1354 & 12.44 & 14.93 & 4.12 & $<$11.22 & B & 60.3 &  & 295 \\
 CT Ori & 06072$+$0953 & 6.16 & 5.57 & 1.22 & $<$1.54 & B & 135.6 &  & 330 \\
 SU Gem & 06108$+$2734 & 7.90 & 5.69 & 2.16 & $<$11.66 & A & 50.1 & b & 575 \\
 UY CMa & 06160$-$1701 & 3.51 & 2.48 & 0.57 & $<$1.00 & B & 113.9 & a & 420 \\
 U Mon  & 07284$-$0940 & 124.30 & 88.43 & 26.28 & 9.24 & A & 92.3 & b & 480 \\
 AR Pup & 08011$-$3627 & 131.33 & 94.32 & 25.81 & 11.65 & B & 75.0 & b & 450 \\
 IW Car & 09256$-$6324 & 101/06 & 96.24 & 34.19 & 13.07 & B & 67.5 & b & 395 \\
 GK Car & 11118$-$5726 & 2.87 & 2.48 & 0.78 & $<$12.13 & B & 55.6 &  & 405 \\
 RU Cen & 12067$-$4508 & 5.36 & 11.02 & 5.57 & 2.01 & B & 64.7 &  & 255 \\
 SX Cen & 12185$-$4856 & 5.95 & 3.62 & 1.09 & $<$1.50 & B & 32.9 & b & 590 \\
 AI Sco & 17530$-$3348 & 17.68 & 11.46 & 2.88 & $<$45.62 & A & 71.0 & b & 480 \\
 AC Her & 18281$+$2149 & 41.47 & 65.33 & 21.12 & 7.79 & B & 75.5 & a & 260 \\
 R Sct  & 18448$-$0545 & 20.88 & 9.30 & 8.10 & $<$138.78 & A & 140.2 & a \\
 R Sge  & 20117$+$1634 & 10.63 & 7.57 & 2.10 & $<$1.66 & A & 70.6 & b & 455 \\
 V Vul  & 20343$+$2625 & 12.39 & 5.72 & 1.29 & $<$6.96 & A & 75.7 & a & 690\\
\hline
\end{tabular}
\end{minipage}
\end{table*}

\section[]{Description of the Envelope\\* Model}

If we assume that the dust grains in the envelope are  predominantly of
the same kind and are in thermal  equilibrium, the luminosity at
frequency $\nu$ in the infrared is given by
\begin{equation}
   L(\nu)=\mskip-12mu\int\limits_{\rmn{envelope}}\mskip-12mu
   \rho(r)Q_{\rmn{abs}}(\nu)B[\nu,T_{\rmn{g}}(r)]\exp [-\tau(\nu,r)]\>
   \rmn{d}V,
\end{equation}
 where
 $Q_{\rmn{abs}}(\nu)$ is the absorption efficiency at frequency $\nu$,
 $\rho(r)$            is the dust grain density,
 $T_{\rmn{g}}(\nu)$    is the grain temperature,
 $B[\nu,T_{\rmn{g}}(r)]$  is the Planck function, and
 $\tau(\nu,r)$        is the optical depth at distance {\it r\/} from the
                      centre of the star.

The temperature $T_{\rmn{g}}(r)$ is determined by the condition of energy
balance: amount of energy radiated = amount of energy absorbed. The
amount of energy absorbed at any point is proportional to the total
available energy at that point, which consists of:
\begin{enumerate}
  \item the attenuated and diluted stellar radiation;
  \item scattered radiation, and
  \item reradiation from other grains.
\end{enumerate}

Detailed solutions of radiative transfer in circumstellar  dust
shells by \citet{b21,b22} indicate that the effect of heating by
other grains becomes significant only at large optical depths at
the absorbing frequencies $[\tau(\rmn{UV})\gg 10]$, and at optical
depths $\tau(\rmn{UV})<1$ the grains have approximately the same
temperature that they would have if they were seeing the starlight
unattenuated and no other radiation.

The Planck mean optical depths of circumstellar envelopes  around
several RV Tauri stars, derived from the ratios of the
luminosities of the dust shell (at infrared wavelengths) and the
star, range from 0.07 to 0.63 \citep{b9}. There is much
uncertainty in the nature of the optical properties of dust grains
in the envelope. The carbon-rich RV Tauri stars are also reported
to show the 10-$\umu$m silicate emission feature typical of
oxygen-rich objects \citep{b6,b19}. The pure terrestrial silicates
or lunar silicates are found to be completely unsuitable to
account for the infrared emission from circumstellar dust shells
around M-type stars \citep{b21}. We assume that the absorption
efficiency $Q_{\rmn{abs}} (\nu)$ in the infrared varies as
$\nu^{\gamma}$. ${\gamma}=1$ appears to provide a reasonable fit
in a variety of sources \citep*{b11,b12}. Under these
circumstances the condition of energy balance implies that the
dust temperature $T_{\rmn{g}}$ will vary as $r^{\beta}$.

In view of the low value of the observed Planck mean  optical depth for
the stellar radiation and the nature of the assumed frequency
dependence of the absorption efficiency, the extinction of the infrared
radiation by the dust envelope can be neglected. If we consider the
envelope to be spherically symmetric, equation (1) reduces to
\begin{equation}
   L(\nu)=\!\!\int_{r_{1}}^{r_{2}}\!\!4\upi r^2\rho(r)\> Q_{\rmn{abs}}(\nu)B[\nu,T_{\rmn{g}}(r)]\> {\rmn{d}}r,
\end{equation}
where $r_1$ and $r_2$ are the inner and outer radii of the shell. For
a dusty density distribution $\rho(r)\propto r^{\alpha}$ and $r_2\gg
r_1$, equation (2) reduces to
\begin{equation}
   L(\nu)\propto \nu^{2+\gamma-Q}\int_{X_0}^{\infty}{{x^Q}\over
   {\rmn{e}^x-1}}\rmn{d}x ,
\end{equation}
where $Q=-(\alpha+\beta+3)/\beta$ and $X_0=(h\nu /kT_0)$. $T_0$
represents the temperature at the inner boundary of the dust shell
where grains start condensing. In a steady radiation pressure
driven mass outflow in the optically thin case, values of $\alpha$
lie near $-2$ \citep{b8}. $\gamma$ and $\beta$ are related by
$\beta=-2/(\gamma+4)$.

In the {\it IRAS\/} Point Source Catalog \citep[PSC,][]{b2}, the
flux densities have been quoted at the effective wavelengths 12,
25, 60 and \hbox{100\,$\umu$m}, assuming a flat energy spectrum
$[\nu F(\nu)=1]$ for the observed sources. For each model given by
equation (3), using the relative system response, the
colour-correction factors \citep{b3} in each of the {\it IRAS\/}
passbands were calculated and the fluxes were converted into flux
densities expected for a flat energy distribution, as assumed in
the {\it IRAS\/} PSC, so that the computed colours can be directly
compared with the colours determined from the catalogue
quantities. Such a procedure is more appropriate than correcting
the {\it IRAS\/} colours for the energy distribution given by a
particular model and then comparing them with those computed by
the model.\footnote{An example of a footnote.}

\subsection{Colour--colour diagram}

The IR colour is defined as
\[
  [\nu_1]-[\nu_2]=-2.5\log [f(\nu_1)/f(\nu_2)],
\]
 where $\nu_1$ and $\nu_2$ are any two wavebands and $f(\nu_1)$
and $f(\nu_2)$ are the corresponding flux  densities assuming a
flat energy spectrum for the source. In Fig.~1, we have plotted
the [25]--[60] colours  of RV Tauri stars against their
corresponding [12]--[25]  colours derived from the {\it IRAS\/}
data. Filled circles  represent stars of group A and open circles
stars of group B. The two sets of near-parallel lines represent
the loci of constant inner shell temperature $T_0$ and the
quantity $Q$ defined above. The models correspond to the case of
absorption efficiency $Q_{\rmn{abs}}(\nu)$ varying as $\nu$ (with
$\gamma=1$ and hence $\beta=-0.4$). We have omitted R Sct in
Fig.~1 because it shows a large deviation from the average
relation shown by all the other objects. R Sct has a comparatively
large excess at 60$\,\umu$m, but the extent of a possible
contamination by the infrared cirrus \citep{b16} is unknown.
\citet{b9} found no evidence of the presence of a dust envelope at
near-IR wavelengths and the spectrum was consistent with a stellar
continuum. This explains why R Sct lies well below the mean
relation shown by stars of groups A and C between the
[3.6]--[11.3] colour excess and the photometrically determined
(Fe/H) \citep{b4}. R Sct has the longest period of 140$\,$d among
the RV Tauri stars detected at far-infrared wavelengths and does
not have the 10-$\umu$m emission feature seen in other objects
\citep{b5,b19}. R Sct is probably the most irregular RV Tauri star
known \citep{b17}.

\begin{figure}
 \vspace{302pt}
 \caption{Plot of [25]--[60] colours of RV  Tauri stars against their
  [12]--[25] colours after  normalizing as indicated in \citet{b3}.
  Some of the objects are identified by their variable-star
  names. Typical error bars are shown in the bottom right-hand corner.
  The lines represent the loci for constant inner shell temperature and
  the quantity $Q$. Note the separation of group A and B stars at $T_0
  \sim$ 460$\,$\,K. Positions occupied by a sample of carbon and oxygen
  Miras are also shown. The $Q=1.0$ line differs from the blackbody line
  by a maximum of $\sim 0.05$.}
\end{figure}
The inner shell temperatures $(T_0)$ derived for the various objects
are also given in Table~1 and we find the majority of them to have
temperatures in the narrow range 400--600$\,$K. If the dependences of
$Q_{\rmn{abs}}(\nu)$ on $\nu$ and $\rho(r)$ on $r$ are similar in all
the objects considered, then in the colour--colour diagram they all
should lie along a line corresponding to different values of $T_0$ and
in Fig.~1 we find that this is essentially the  case. In view of the
quoted uncertainties in the flux measurements, we cannot attach much
significance to the scatter in Fig.~1.

At \hbox{100\,$\umu$m} the infrared sky is characterized by
emission, called infrared cirrus, from interstellar dust on all
spatial scales \citep{b16}, thereby impairing the measurements at
far-infrared wavelengths. In Fig.~2, we have plotted the
[60]--[100] colours of the six RV Tauri stars detected at
\hbox{100\,$\umu$m} against their [25]--[60] colours, along with
the grid showing the regions of different values for inner shell
temperature $T_0$ and the quantity $Q$, as in Fig.~1. The results
indicated by Fig.~2 are consistent with those derived from Fig.~1.
AR Pup shows a large excess at \hbox{100\,$\umu$m} but, in view of
the large values for the cirrus flags given in the catalogue, the
intrinsic flux at \hbox{100\,$\umu$m} is uncertain.

\subsection{Radial distribution of dust}

\begin{figure*}
  \vspace*{174pt}
  \caption{Plot of the [60]--[100] colours of RV Tauri stars against
  their [25]--[60] colours after normalizing as indicated in \citet{b3}.
  The solid lines represent the loci for constant
  inner shell temperature and the quantity $Q$. The dashed line shows
  the locus for a blackbody distribution.}
\end{figure*}

From Fig.~1, it is evident that all RV Tauri stars lie between the
lines corresponding to $Q=1.5$ and 0.5. With
 \[
  \alpha=-(1+Q)\beta-3,
 \]
 these values suggest limits of $r^{-2.0}$ and $r^{-2.4}$ for the
dust density variation, indicating a near-constant mass-loss rate.
\citet{b12} has suggested that the density in the circumstellar
envelope around RV Tauri stars varies as $r^{-1}$, implying a
mass-loss rate that was greater in the past than it is currently.
By fitting a power law to the observed fluxes, such that $f(\nu)$
varies as $\nu^q$, values of $q$ determined by him for the various
objects given in Table~1 lie in the range 0.6--1.2, with a mean
$\skew5\bar q=0.98$. The assumption of a power law corresponds to
the case of $X_0=0$ in equation (3) and hence we get
 \[
  q=2+\gamma -Q.
 \]
Since we assume that $Q_{\rmn{abs}}(\nu)$ varies as $\nu$, the
resulting value for $Q$=2.0. None of the objects is found to lie in the
corresponding region in the colour--colour diagram. Even this extreme
value for $Q$ implies a density which varies as $r^{-1.8}$.

\citet{b9} have reported that the simultaneous optical and near-IR
data of AC Her can be fitted by a combination of two blackbodies
at 5680 and 1800\,K, representing, respectively, the stellar and
dust shell temperatures, and suggested that in RV Tauri stars the
grain formation is a sporadic phenomenon and not a continuous
process. Apparently, they have been influenced by the remark by
\citet{b7} that their data in the 3.5--11$\,\umu$m region of AC
Her indicated a dust temperature of $\sim$300\,K. We find that the
{\it K--L\/} colours given by \citet{b5}, \citet{b15} and
\citet{b9} are all consistent with each other. Surely, hot dust
($\sim 1800\,$K), if present at the time of observations by
\citet{b9}, would have affected the {\it K--L\/} colour
significantly. AC Her, like other members of its class, is found
to execute elongated loops in the ({\it U--B\/}), ({\it B--V\/})
plane \citep{b20}, indicating that significant departure of the
stellar continuum from the blackbody is to be expected. Further,
their data show only a marginal excess at the near-IR wavelengths.
We feel that the case for the existence of hot dust around AC Her
and hence for the sporadic grain formation around RV Tauri stars
is not strong. In Fig.~3 we find that AC Her and RU Cen lie very
close to R Sct which, according to \citet{b9}, shows no evidence
for the presence of a hot dust envelope.

\subsubsection{Comparison with oxygen and carbon Miras}

In Fig.~1 we have also shown the positions of a sample of
oxygen-rich and carbon-rich Miras. At the low temperatures
characteristic of the Miras, a part of the emission at 12$\,\umu$m
comes from the photosphere. For a blackbody at 2000$\,$K, the
ratio of fluxes at wavelengths of 12 and 2$\,\umu$m
$(f_{12}/f_{2})\sim 0.18$. The Miras shown in Fig.~1 have
$(f_{12}/f_{2})$ ratios larger than twice the above value. It is
clear that the three groups of objects populate three different
regions of the diagram. \citet{b10} have already noticed that
there are distinct differences between the {\it IRAS\/} colours of
oxygen-rich and carbon-rich objects. On the basis of an analysis,
using a bigger sample of bright giant stars in the {\it IRAS\/}
catalogue, this has been interpreted by \citet{b25} as being due
to a systematic difference in the dust grain emissivity index. U
Mon shows the 10-$\umu$m silicate emission convincingly and, in
most of the other objects for which low-resolution spectra in the
near-infrared have been reported \citep{b5,b19}, the 10-$\umu$m
emission may be partly attributed to silicates. Hence it is
reasonable to expect that, in the envelopes around at least some
of the RV Tauri stars, the dust grains are predominantly of
silicates, as in the case of oxygen Miras \citep{b21}. The fact
that none of the RV Tauri stars is found in the region of the
two-colour diagram occupied by the oxygen Miras indicates that the
emissivity indices of the silicate grains in the two cases are
different. Because of the higher temperatures and luminosities,
the environment of grain formation will be different in RV Tauri
stars.

\subsubsection{Correlation with subgroups}

\citet{b20} have identified three spectroscopic subgroups, which
are designated as groups A, B and C. Objects of group A are
metal-rich; group C are metal-poor; group B objects are also
metal-poor, but  show carbon enhancements \citep{b20,b14,b4,b1}.
It is interesting to see that Table~1 contains no group C objects
and that in Fig.~1 there is a clear separation of the two
spectroscopic subgroups A and B, with the demarcation  occurring
at an inner shell temperature of about 450$\,$K, group B stars
having lower temperatures than group A. SX Cen is the only
exception. \citet{b14} has reported that metal lines are stronger
in SX Cen than in other group B objects. It may be worth noting
that SX Cen has the shortest period among the 100 or so objects
with the RV Tauri classification. RU Cen has the coolest inner
shell temperature, as already suggested by the near-infrared
spectrum \citep{b6}.
\begin{figure}
  \vspace*{174pt}
  \caption{Plot of ({\it K--L\/}) colours of RV Tauri stars detected by
  {\it IRAS\/} against their corresponding ({\it J--K\/}) colours. The
  position of AR Pup is indicated. The three objects lying close to the
  blackbody line are AC Her, RU Cen and R Sct.}
\end{figure}

Group B objects follow a different mean relationship from those of
group A, having systematically larger 11-$\umu$m excess for a
given excess at 3$\,\umu$m \citep{b15}. For a general sample of RV
Tauri stars, the distinction between the oxygen-rich and
carbon-rich objects is not that apparent in the {\it JHKL\/}
bands. In Fig.~3 we have plotted the near-IR magnitudes of the
objects given in Table~1 (except V Vul which has no available
measurements) in the {\it J--K, K--L\/} plane. The colours,  taken
from \citet{b15} and \citet{b9}, are averaged if more than one
observation exists, because the internal agreements are found to
be often of the order of observational uncertainties, in
accordance with the earlier finding by \citet{b5} that variability
has relatively little effect on colours. Barring RU Cen and AC
Her, it is evident that stars belonging to group B show
systematically larger excesses at {\it L\/} band for a given
excess at {\it K}. The low excesses at near-IR wavelengths for AC
Her and RU Cen are consistent with the very low dust temperatures
indicated by the far-infrared colours.
%
\begin{figure*}
\vbox to 220mm{\vfil Landscape figure to go here. This figure was
not part of the original paper and is inserted here for
illustrative purposes.\\ See the author guide for details (section
2.2 of \verb|mn2eguide.tex|) on how to handle landscape figures or
tables. \caption{} \vfil} \label{landfig}
\end{figure*}

It is already well established that from {\it UBV\/} photometry
one can distinguish between groups A and B,  members of group A
being significantly redder than those of group B \citep{b20}.
Similarly, \citet{b4} has found that the two spectroscopic groups
are well separated in the DDO colour--colour diagrams when mean
colours are used for the individual objects.

The clear separation of the spectroscopic subgroups A and  B in
the IR two-colour diagram suggests that the natures of dust grains
in the envelopes in the two cases are not  identical. This is to
be expected because of the differences in the physical properties
of the stars themselves. The average colours of group B stars are
bluer than group A, but the envelope dust temperatures of B are
cooler than those of A. The near-IR spectra of AC Her and RU Cen
are extremely similar \citep{b6}. The striking similarities in the
optical spectra of AC Her and RU Cen have been pointed out by
Bidelman \citep{b18}. We feel that the physical properties,
including the chemical composition, of the grains  formed in the
circumstellar envelope strongly depend on those of the embedded
star. This, probably, explains the diversity of the energy
distributions of RV Tauri stars in the near-infrared found by
\citet{b6}. On the basis of the observed differences in chemical
abundances and space distribution of RV Tauri stars, \citet{b15}
has already pointed out that there is no direct evolutionary
connection between group A and group B objects, thus ruling out
the possibility that group B objects are the evolutionary
successors of group A, in which grain formation has stopped and
the cooler temperatures for the former are caused by an envelope
expansion.

\citet{b13} have subdivided RV Tauri stars into two classes, RVa
and RVb, on the basis of their light curves; the former shows a
constant mean  brightness, whereas the latter shows a cyclically
varying  mean brightness. Extensive observations in the
near-infrared show that, on average, RVb stars are redder than RVa
stars, and \citet{b15} has suggested that in RVb stars dust shells
are denser in the inner regions and hence radiate strongly in the
1--3$\,\umu$m region. Fig.~3 confirms this; RVb objects show
systematically larger ({\it J--K\/}) and ({\it K--L\/}) colours
than RVa objects. Apparently, there is no distinction between
objects of the two light-curve types at far-infrared wavelengths
(Fig.~1).

\section{Conclusions}

In the [12]--[25], [25]--[60] colour diagram, RV Tauri stars populate
cooler temperature regions $(T<600 \,\rmn{K})$, distinctly different from
those occupied by the oxygen and carbon Miras. Using a simple model
in which
\begin{enumerate}
  \item the envelope is spherically symmetric,
  \item the IR-emitting grains are predominantly of the same kind, and
  \item in the infrared the absorption efficiency $Q_{\rmn{abs}}
        (\nu)\propto\nu$,
\end{enumerate}
we find that the {\it IRAS\/} fluxes are consistent with the
density in the envelope $\rho(r)\propto r^{-2}$, where {\it r\/}
is the radial distance. Such a dependence for the dust density
implies that the mass-loss rates in RV Tauri stars have not
reduced considerably during the recent past, contrary to the
suggestion by \citet{b12}. In the two-colour diagram, the
blackbody line and the line corresponding to $\rho(r)\propto
r^{-2.2}$ nearly overlap and the present data are insufficient to
resolve between the two cases. The latter case is more physically
reasonable, however.

The spectroscopic subgroups A and B are well separated in  the
{\it IRAS\/} two-colour diagram, with group B objects  having
systematically cooler dust envelopes. If we consider only the
objects detected by {\it IRAS}, we find that stars belonging to
group B show systematically larger excess at {\it L\/} band for a
given excess at {\it K}. Apparently, there is no correlation
between the light-curve types (RVa and RVb) and the far-infrared
behaviour of these objects. It is fairly certain that the physical
properties, including the chemical composition, of the embedded
stars are directly reflected by those of the dust grains. Most
probably, the grain formation process in RV Tauri stars is
continuous and not sporadic as suggested by \citet{b9}.

\section*{Acknowledgments}

I thank Professor N. Kameswara Rao for some helpful suggestions,
Dr H. C. Bhatt for a critical reading of the original version of the
paper and an anonymous referee for very useful comments that improved
the presentation of the paper.


\begin{thebibliography}{99}
\bibitem[\protect\citeauthoryear{Baird}{1981}]{b1} Baird S.R., 1981,
ApJ, 245, 208
\bibitem[\protect\citeauthoryear{Beichman et al.}{1985a}]{b2} Beichman
C.A., Neugebauer G., Habing H.J., Clegg P.E., Chester T.J., 1985a,
{\it IRAS\/} Point Source Catalog. Jet Propulsion Laboratory,
Pasadena
\bibitem[\protect\citeauthoryear{Beichman et al.}{1985b}]{b3} Beichman
C.A., Neugebauer G., Habing H.J., Clegg P.E., Chester T.J., 1985b,
{\it IRAS\/} Explanatory Supplement. Jet Propulsion Laboratory,
Pasadena
\bibitem[\protect\citeauthoryear{Dawson}{1979}]{b4} Dawson D.W., 1979,
ApJS, 41, 97
\bibitem[\protect\citeauthoryear{Gerhz}{1972}]{b5} Gerhz R.D., 1972, ApJ,
178, 715
\bibitem[\protect\citeauthoryear{Gerhz \& Ney}{1972}]{b6} Gerhz R.D., Ney
E.P., 1972, PASP, 84, 768
\bibitem[\protect\citeauthoryear{Gerhz \& Woolf}{1970}]{b7} Gerhz R.D., Woolf N.J.,
1970, ApJ, 161, L213
\bibitem[\protect\citeauthoryear{Gilman}{1972}]{b8} Gilman R.C., 1972, ApJ, 178, 423
\bibitem[\protect\citeauthoryear{Goldsmith et al.}{1987}]{b9} Goldsmith M.J., Evans A.,
Albinson J.S., Bode M.F., 1987, MNRAS, 227, 143
\bibitem[\protect\citeauthoryear{Hacking et al.}{1985}]{b10} Hacking P. et al., 1985,
PASP, 97, 616
\bibitem[\protect\citeauthoryear{Harvey, Thronson \& Gatley}{Harvey et al.}{1979}]{b11}
Harvey P.M., Thronson H.A., Gatley I., 1979, ApJ, 231, 115
\bibitem[\protect\citeauthoryear{Jura}{1986}]{b12} Jura M., 1986, ApJ, 309, 732
\bibitem[\protect\citeauthoryear{Kukarkin et al.}{1969}]{b13} Kukarkin B.V. et al.,
1969, General Catalogue of Variable Stars. Moscow
\bibitem[\protect\citeauthoryear{Lloyd Evans}{1974}]{b14} Lloyd Evans T., 1974, MNRAS,
167, 17{\sc p}
\bibitem[\protect\citeauthoryear{Lloyd Evans}{1985}]{b15} Lloyd Evans T., 1985, MNRAS,
217, 493
\bibitem[\protect\citeauthoryear{Low et al.}{1984}]{b16} Low F.J. et al., 1984, ApJ,
278, L19
\bibitem[\protect\citeauthoryear{McLaughlin}{1932}]{b17} McLaughlin D.B., 1932, Publ. Univ.
Obs. Mich., 4, 135
\bibitem[\protect\citeauthoryear{O'Connell}{1961}]{b18} O'Connell J.K., 1961, Specola
Vaticana Ric. Astron., 6, 341
\bibitem[\protect\citeauthoryear{Olnon \& Raimond}{1986}]{b19} Olnon F.M., Raimond E., 1986,
A\&AS, 65, 607
\bibitem[\protect\citeauthoryear{Preston et al.}{1963}]{b20} Preston G.W., Krzeminski W., Smak J.,
Williams J.A., 1963, ApJ, 137, 401
\bibitem[\protect\citeauthoryear{Rowan-Robinson \& Harris}{1983a}]{b21} Rowan-Robinson M., Harris
S., 1983a, MNRAS, 202, 767
\bibitem[\protect\citeauthoryear{Rowan-Robinson \& Harris}{1983b}]{b22} Rowan-Robinson M., Harris
S., 1983b, MNRAS, 202, 797
\bibitem[\protect\citeauthoryear{van der Veen \& Habing}{1988}]{b23} van der Veen W.E.C.J., Habing
H.J., 1988, A\&A, 194, 125
\bibitem[\protect\citeauthoryear{Willems \& de Jong}{1988}]{b24} Willems F.J., de Jong T., 1988,
A\&A, 196, 173
\bibitem[\protect\citeauthoryear{Zuckerman \& Dyck}{1986}]{b25} Zuckerman B., Dyck H.M., 1986, ApJ,
311, 345
\end{thebibliography}

\appendix

\section[]{Large gaps in L\lowercase{y}${\balpha}$ forests\\* due to fluctuations in line distribution}

(This appendix was not part of the original paper by
A.V.~Raveendran and is included here just for illustrative
purposes. The references are not relevant to the text of the
appendix, they are references from the bibliography used to
illustrate text before and after citations.)

Spectroscopic observations of bright quasars show that the mean
number density of Ly$\alpha$ forest lines, which satisfy certain
criteria, evolves like $\rmn{d}N/\rmn{d}z=A(1+z)^\gamma$, where
$A$ and~$\gamma$ are two constants.  Given the above intrinsic
line distribution we examine the probability of finding large gaps
in the Ly$\alpha$ forests.  We concentrate here only on the
statistics and neglect all observational complications such as the
line blending effect \citep[see][for example]{b11}.

Suppose we have observed a Ly$\alpha$ forest between redshifts $z_1$
and~$z_2$ and found $N-1$ lines.  For high-redshift quasars $z_2$~is
usually the emission redshift $z_{\rmn{em}}$ and $z_1$ is set to
$(\lambda_{\rmn{Ly}\beta}/\lambda_{\rmn{Ly}\alpha})(1+z_{\rmn{em}})=0.844
(1+z_{\rmn{em}})$ to avoid contamination by Ly$\beta$ lines.  We
want to know whether the largest gaps observed in the forest are
significantly inconsistent with the above line distribution.  To do
this we introduce a new variable~$x$:
%
\begin{equation}
x={(1+z)^{\gamma+1}-(1+z_1)^{\gamma+1} \over
     (1+z_2)^{\gamma+1}-(1+z_1)^{\gamma+1}}.
\end{equation}
%
$x$ varies from 0 to 1.  We then have $\rmn{d}N/\rmn{d}x=\lambda$, where $\lambda$
is the mean number of lines between $z_1$ and $z_2$ and is given by
%
\begin{equation}
\lambda\equiv{A[(1+z_2)^{\gamma+1}-(1+z_1)^{\gamma+1}]\over\gamma+1}.
\end{equation}
%
This means that the Ly$\alpha$ forest lines are uniformly
distributed in~$x$. The probability of finding $N-1$ lines between $z_1$
and~$z_2$, $P_{N-1}$, is assumed to be the Poisson distribution.
%
\newpage
%
\begin{figure}
\vspace{11pc}
\caption{$P(>x_{\rmn{gap}})$ as a function of $x_{\rmn{gap}}$ for,
 from left to right, $N=160$, 150, 140, 110, 100, 90, 50, 45 and~40.
 Compare this with \protect\citet{b15}.}
\label{appenfig}
\end{figure}

\subsection{Subsection title}

We plot in Fig.~\ref{appenfig} $P(>x_{\rmn{gap}})$ for several $N$
values. We see that, for $N=100$ and $x_{\rmn{gap}}=0.06$,
$P(>0.06)\approx20$ per cent.  This means that the probability of
finding a gap with a size larger than six times the mean
separation is not significantly small. When the mean number of
lines is large, $\lambda\sim N>>1$, our $P(>x_{\rmn{gap}})$
approaches the result obtained by \citet[fig. 4]{b22} for small
(but still very large if measured in units of the mean separation)
$x_{\rmn{gap}}$, i.e., $P(>x_{\rmn{gap}})\sim N(1-
x_{\rmn{gap}})^{N-1}\sim N {\rmn{exp}}(-\lambda x_{\rmn{gap}})$.

\bsp

\label{lastpage}

\end{document}
